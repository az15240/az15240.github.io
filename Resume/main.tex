\documentclass[10pt, letterpaper]{article}

% Packages:
\usepackage[
    ignoreheadfoot, % set margins without considering header and footer
    top=2 cm, % seperation between body and page edge from the top
    bottom=2 cm, % seperation between body and page edge from the bottom
    left=2 cm, % seperation between body and page edge from the left
    right=2 cm, % seperation between body and page edge from the right
    footskip=1.0 cm, % seperation between body and footer
    % showframe % for debugging 
]{geometry} % for adjusting page geometry
\usepackage{titlesec} % for customizing section titles
\usepackage{tabularx} % for making tables with fixed width columns
\usepackage{array} % tabularx requires this
\usepackage[dvipsnames]{xcolor} % for coloring text
\definecolor{primaryColor}{RGB}{0, 79, 144} % define primary color
\usepackage{enumitem} % for customizing lists
\usepackage{fontawesome5} % for using icons
\usepackage{amsmath} % for math
\usepackage[
    pdftitle={Stanley Yang's CV},
    pdfauthor={Stanley Yang},
    pdfcreator={LaTeX with RenderCV},
    colorlinks=true,
    urlcolor=primaryColor
]{hyperref} % for links, metadata and bookmarks
\usepackage[pscoord]{eso-pic} % for floating text on the page
\usepackage{calc} % for calculating lengths
\usepackage{bookmark} % for bookmarks
% \usepackage{lastpage} % for getting the total number of pages
\usepackage{changepage} % for one column entries (adjustwidth environment)
\usepackage{paracol} % for two and three column entries
\usepackage{ifthen} % for conditional statements
\usepackage{needspace} % for avoiding page brake right after the section title
\usepackage{iftex} % check if engine is pdflatex, xetex or luatex

% Ensure that generate pdf is machine readable/ATS parsable:
\ifPDFTeX
    \input{glyphtounicode}
    \pdfgentounicode=1
    % \usepackage[T1]{fontenc} % this breaks sb2nov
    \usepackage[utf8]{inputenc}
    \usepackage{lmodern}
\fi



% Some settings:
\AtBeginEnvironment{adjustwidth}{\partopsep0pt} % remove space before adjustwidth environment
\pagestyle{empty} % no header or footer
\setcounter{secnumdepth}{0} % no section numbering
\setlength{\parindent}{0pt} % no indentation
\setlength{\topskip}{0pt} % no top skip
\setlength{\columnsep}{0cm} % set column seperation
\makeatletter
\let\ps@customFooterStyle\ps@plain % Copy the plain style to customFooterStyle
% \patchcmd{\ps@customFooterStyle}{\thepage}{
    % \color{gray}\textit{\small John Doe - Page \thepage{} of \pageref*{LastPage}}
% }{}{} % replace number by desired string
\makeatother
% \pagestyle{customFooterStyle}

\titleformat{\section}{\needspace{4\baselineskip}\bfseries\large}{}{0pt}{}[\vspace{1pt}\titlerule]

\titlespacing{\section}{
    % left space:
    -1pt
}{
    % top space:
    0.3 cm
}{
    % bottom space:
    0.2 cm
} % section title spacing

\renewcommand\labelitemi{$\circ$} % custom bullet points
\newenvironment{highlights}{
    \begin{itemize}[
        topsep=0.1 cm,
        parsep=0.10 cm,
        partopsep=0pt,
        itemsep=0pt,
        leftmargin=0.4 cm + 10pt
    ]
}{
    \end{itemize}
} % new environment for highlights

\newenvironment{highlightsforbulletentries}{
    \begin{itemize}[
        topsep=0.10 cm,
        parsep=0.10 cm,
        partopsep=0pt,
        itemsep=0pt,
        leftmargin=10pt
    ]
}{
    \end{itemize}
} % new environment for highlights for bullet entries


\newenvironment{onecolentry}{
    \begin{adjustwidth}{
        0 cm + 0.00001 cm
    }{
        0.1 cm + 0.00001 cm
    }
}{
    \end{adjustwidth}
} % new environment for one column entries

\newenvironment{twocolentry}[2][]{
    \onecolentry
    \def\secondColumn{#2}
    \setcolumnwidth{\fill, 4.5 cm}
    \begin{paracol}{2}
}{
    \switchcolumn \raggedleft \secondColumn
    \end{paracol}
    \endonecolentry
} % new environment for two column entries

\newenvironment{header}{
    \setlength{\topsep}{0pt}\par\kern\topsep\centering\linespread{1.5}
}{
    \par\kern\topsep
} % new environment for the header

% \newcommand{\placelastupdatedtext}{% \placetextbox{<horizontal pos>}{<vertical pos>}{<stuff>}
%   \AddToShipoutPictureFG*{% Add <stuff> to current page foreground
%     \put(
%         \LenToUnit{\paperwidth-2 cm-0.2 cm+0.05cm},
%         \LenToUnit{\paperheight-1.0 cm}
%     ){\vtop{{\null}\makebox[0pt][c]{
%         \small\color{gray}\textit{Last updated in September 2024}\hspace{\widthof{Last updated in September 2024}}
%     }}}%
%   }%
% }%

% save the original href command in a new command:
\let\hrefWithoutArrow\href

% new command for external links:
\renewcommand{\href}[2]{\hrefWithoutArrow{#1}{\ifthenelse{\equal{#2}{}}{ }{#2 }\raisebox{.15ex}{\footnotesize \faExternalLink*}}}


\begin{document}
    \newcommand{\AND}{\unskip
        \cleaders\copy\ANDbox\hskip\wd\ANDbox
        \ignorespaces
    }
    \newsavebox\ANDbox
    \sbox\ANDbox{}

    % \placelastupdatedtext
    \begin{header}
        \textbf{\fontsize{20 pt}{24 pt}\selectfont Stanley Yang}

        \vspace{0.3 cm}

        \normalsize
        % \mbox{{\color{black}\footnotesize\faMapMarker*}\hspace*{0.13cm}Seattle, WA}%
        % \kern 0.20 cm%
        % \AND%
        % \kern 0.25 cm%
        \mbox{\hrefWithoutArrow{mailto:guangyg@cs.washington.edu}{\color{black}{\footnotesize\faEnvelope[regular]}\hspace*{0.13cm}guangyg@cs.washington.edu}}%
        \kern 0.25 cm%
        % \AND%
        % \kern 0.25 cm%
        % \mbox{\hrefWithoutArrow{tel:+90-541-999-99-99}{\color{black}{\footnotesize\faPhone*}\hspace*{0.13cm}0541 999 99 99}}%
        % \kern 0.25 cm%
        \AND%
        \kern 0.25 cm%
        \mbox{\hrefWithoutArrow{https://az15240.github.io/}{\color{black}{\footnotesize\faLink}\hspace*{0.13cm}az15240.github.io}}%
        \kern 0.25 cm%
        \AND%
        \kern 0.25 cm%
        \mbox{\hrefWithoutArrow{https://linkedin.com/in/stanley-yang-9457b7252}{\color{black}{\footnotesize\faLinkedinIn}\hspace*{0.13cm}stanley-yang-9457b7252}}%
        % \kern 0.25 cm%
        % \AND%
        % \kern 0.25 cm%
        % \mbox{\hrefWithoutArrow{https://github.com/az15240}{\color{black}{\footnotesize\faGithub}\hspace*{0.13cm}az15240}}%
    \end{header}

    \vspace{0.3 cm - 0.3 cm}


    % \section{Welcome to RenderCV!}


    %     \begin{onecolentry}
    %         \href{https://rendercv.com}{RenderCV} is a LaTeX-based CV/resume version-control and maintenance app. It allows you to create a high-quality CV or resume as a PDF file from a YAML file, with \textbf{Markdown syntax support} and \textbf{complete control over the LaTeX code}.
    %     \end{onecolentry}

    %     \vspace{0.2 cm}

    %     \begin{onecolentry}
    %         The boilerplate content was inspired by \href{https://github.com/dnl-blkv/mcdowell-cv}{Gayle McDowell}.
    %     \end{onecolentry}
    
    % \section{Quick Guide}

    % \begin{onecolentry}
    %     \begin{highlightsforbulletentries}


    %     \item Each section title is arbitrary and each section contains a list of entries.

    %     \item There are 7 unique entry types: \textit{BulletEntry}, \textit{TextEntry}, \textit{EducationEntry}, \textit{ExperienceEntry}, \textit{NormalEntry}, \textit{PublicationEntry}, and \textit{OneLineEntry}.

    %     \item Select a section title, pick an entry type, and start writing your section!

    %     \item \href{https://docs.rendercv.com/user_guide/}{Here}, you can find a comprehensive user guide for RenderCV.


    %     \end{highlightsforbulletentries}
    % \end{onecolentry}

    \section{Education}



        
        \begin{twocolentry}{
            
            
        \textit{Sept 2022 - June 2026}}
            \textbf{University of Washington}, Seattle, WA

            \textit{Bachelor of Science in Computer Science}, Major GPA 3.92/4.00
        \end{twocolentry}

        \vspace{0.10 cm}
        \begin{onecolentry}
            \begin{highlights}
                \item \textbf{Coursework:} Software Design, Data Structure, Database, Machine Learning, Two-Year Honor Math Series
                \item \textbf{Award:} UW ICPC Winter Programming Contest 2024 - Second Place
            \end{highlights}
        \end{onecolentry}



    
    \section{Experience}



        \begin{twocolentry}{
        \textit{Shanghai, China}    
            
        \textit{June 2024 – Sept 2024}}
            \textbf{\href{https://dgl.ai/}{Applied Scientist Intern}}
            
            \textit{Amazon AI Lab}
        \end{twocolentry}

        \vspace{0.10 cm}
        \begin{onecolentry}
            \begin{highlights}
\item	Enhanced \textbf{Deep Graph Library} with bug fixes, performance optimizations, and automated pipelines
\item	Implemented reverse edge feature to graph training datasets, boosting node classification accuracy by \textbf{16\%}
\item	Improved \textbf{CSC graph neighbor sampling efficiency by 6.5\%} via backend \textbf{PyTorch} operator optimization 
\item	Built a \textbf{Docker-based release pipeline}, incorporating \textbf{unit tests} and \textbf{daily regression framework}
\item	Integrated \textbf{version update automation} and \textbf{AWS S3 deployment} for efficient \textbf{wheel distribution}

            \end{highlights}
        \end{onecolentry}


        \vspace{0.2 cm}

        \begin{twocolentry}{
        \textit{Seattle, WA}    
            
        \textit{March 2023 – June 2024}}
            \textbf{\href{https://cse.uw.edu/341}{Teaching Assistant}}
            
            \textit{Paul G. Allen School of Computer Science \& Engineering}
        \end{twocolentry}

        \vspace{0.10 cm}
        \begin{onecolentry}
            \begin{highlights}
\item	Led course on \textbf{functional programming}, type systems and interpreter design using \textbf{OCaml} and \textbf{Racket}
\item	Developed \textbf{autograder scripts} with \textbf{700+ test cases} and led \textbf{infrastructure development}
\item	Assisted professors in homework design, created rubrics, and coordinated TA grading for \textbf{600+ assignments }
\item \textbf{Co-taught a guest lecture} on ``Static vs. Dynamic Typing" with head TA
            \end{highlights}
        \end{onecolentry}

%%%%%

        \vspace{0.2 cm}

        \begin{twocolentry}{
        \textit{Seattle, WA}    
            
        \textit{June 2023 – Aug 2024}}
            \textbf{\href{https://uwplse.org/}{Database Research Assistant}}
            
            \textit{UW PLSE (Programming Languages and Software Engineering) Lab}
        \end{twocolentry}

        \vspace{0.10 cm}
        \begin{onecolentry}
            \begin{highlights}
\item Optimized processing of \textbf{400+ million} data points using \textbf{SQLite}, executing \textbf{16,000+} queries 
\item Developed \textbf{automated pipeline} using \textbf{bash scripts} to streamline query analysis and data cleaning processes
\item Investigated SQL table equivalences, inspiring a popular \textbf{blog post} with \textbf{272 upvotes on Hacker News}
\item Analyzed SQL null-value handling, proposing a ``column normal form" to \textbf{mitigate }\textbf{unintended side effects}
            \end{highlights}
        \end{onecolentry}


    

    
    \section{Projects}


        \begin{twocolentry}{
            
            
        \textit{Jan 2024 - March 2024}}
            \textbf{CaCL (Change and Chance Language) Interpreter \& Compiler}
        \end{twocolentry}

        \vspace{0.10 cm}
        \begin{onecolentry}
            \begin{highlights}
\item Built a comprehensive interpreter supporting template expansions, mutations, and diverse data types
\item Implemented parallel let expressions and boolean shortcuts to enhance efficiency and logic flow
\item Added support for \textbf{reverse-mode automatic differentiation}, essential for \textbf{machine learning} applications
\item Integrated \textbf{probability distributions and sampling methods} for advanced statistical modeling
\item Authored \textbf{1,300+ lines of tests} to validate functionality and ensure robust error-handling
            \end{highlights}
        \end{onecolentry}


\vspace{0.2 cm}
        
        \begin{twocolentry}{
            
            
        \textit{Jan 2024 - June 2024}}
            \textbf{\href{https://github.com/merlinyx/primtag}{Primitive Tagging for Everyday Objects Research}}
        \end{twocolentry}

        \vspace{0.10 cm}
        \begin{onecolentry}
            \begin{highlights}

\item	Developed semi-automatic methods to identify 3D geometric primitive types and parameters on input meshes
\item	Implemented \textbf{user interface} and cropping functionality for intuitive region selection of key parts
\item	Applied \textbf{differential 3D learning techniques} to optimize primitive shape parameters using \textbf{PyTorch}
\item   Contributed to enhancing FabHacks, a design and visualization system for creating functional assemblies


            \end{highlights}
        \end{onecolentry}




        % \vspace{0.2 cm}
        % \begin{twocolentry}{
%         \textit{Feb 2023 - March 2023}}
%             \textbf{Campus Path Finder}
%         \end{twocolentry}

%         \vspace{0.10 cm}
%         \begin{onecolentry}
%             \begin{highlights}
% \item	Created a web app using Java, React, and Spark framework to navigate 52 campus buildings
% \item	Implemented Dijkstra's algorithm for navigation and MVC pattern for GUI, tested with 5000+ lines of JUnit tests
%             \end{highlights}
%         \end{onecolentry}


    
    \section{Skills}

        
        \begin{onecolentry}
            \textbf{Languages:} Java, C/C++, Python, Shell, JavaScript, SQL, OCaml, Racket, Ruby, LaTeX, MATLAB
        \end{onecolentry}

        \vspace{0.2 cm}

        \begin{onecolentry}
            \textbf{Frameworks:} PyTorch, NumPy, Docker, JUnit, ReactJS, Java Spark, Java Swing, DGL, Figma, AWS, DuckDB
        \end{onecolentry}


    

\end{document}